\section{Blockchain}
A blockchain is a decentralized and distributed digital ledger that records all transactions\cite{moneropedia}. This core technology that makes cryptocurrency possible was originally proposed in 2008 for ``Bitcoin'' by a person or group under pseudonym ``Satoshi Nakamoto''\cite{bitcoin_whitepaper}. The blockchain is stored on a network of nodes (computers running the blockchain's daemon\footnote{A \textbf{Daemon} is a program which runs as a background service and communicates with other nodes.} software) with new transactions and blocks being propagated via a peer-to-peer (P2P) protocol\cite{ledger_transactions}. An LMDB (Lightning Memory-Mapped Database) is used for both Monero and Wownero blockchain data storage\cite{monero_repo,wowrepo}.

% ------------------------------------------ %
\subsection{Transaction Process}
When a transaction is initiated it is broadcast to the network and stored in each node's transaction cache, also known as a ``memory pool,'' if the transaction passes validation checks\cite{ledger_transactions}. These transaction validation checks typically check factors such as confirming the money exists, it has only been sent once, it is being sent by the address which owns it, the transaction received amount equals the input, and the transaction is correctly formatted\cite{zero2monero}.
``Miner'' nodes attempt to create a new ``block'' from this transaction queue and when successful a new block is added to the blockchain\cite{ledger_transactions} containing a list of hashed\footnote{\textbf{``Hashing''} refers to the use of a mathematical algorithm to produce a numeric value that is representative of input data\cite{glossary}.} transactions.

Most blockchains enforce a multiple confirmation requirement before received funds can be used in a transaction again (``unlock time'')\cite{wikipedia_crypto}. This prevents double-spend and similar attacks by making the transaction more difficult to reverse as an attacker would need to overwrite multiple blocks on the blockchain. This becomes exponentially more difficult to do as more confirmations are added. Monero requires 10 confirmations\cite{monero_transactiontime} before funds are unlocked while Wownero requires only 4 confirmations\cite{wowrepo}. For CryptoNight protocol blockchains, like Monero and Wownero, it also serves to prevent people from making multiple transactions at the same block height which may compromise ring signature based transaction obfuscation.

It should be noted that for small transactions it may not be necessary to wait for the full blockchain confirmations and in some cases just the validated initial broadcast may be sufficient to approve a sale.

% ------------------------------------------ %
\subsection{Blocks}
Blocks are created by miner nodes and the ``block time'' (average time interval between block creation) is controlled by adjusting the proof-of-work algorithm difficulty against the hashrate\footnote{\textbf{Hashrate} is a measure of the computational power of a cryptocurrency network.} of the network\cite{randomx_repo,wikipedia_crypto}. The \emph{average} block time for Monero is pre-set at 2 minutes\cite{monero_about} as opposed to 5 minutes for Wownero\cite{wowrepo}.

In addition to a list of hashed transactions, each block will contain header information and a new hashed ``coinbase transaction'' which rewards the miner for creating the block\cite{monero_blocks, ledger_transactions}. The block header data will include a hash of the previous block ensuring that the blockchain must remain a sequential and complete ``chain''\cite{zero2monero,wikipedia_crypto}. This ensures that previous block alteration or removal is impossible without replacement of all subsequent blocks\cite{zero2monero}.

Because it is possible for more transactions to occur between blocks than would fit within the block size limit,\footnote{\textbf{Block size limit} is a pre-set but not hard-coded parameter\cite{CryptoNote}. Note: Both Monero and Wownero have an adaptive block size limit calculated dynamically\cite{monero_transactiontime}.} transactions may remain in the queue until a later block. To prevent transaction spam, Monero has a dynamically calculated minimum fee on all transactions, however, if a sender needs to ensure their transaction is added to the earliest possible block they can raise the fee above the minimum\cite{zero2monero}. Because the transaction fee is given to the miner who creates the block, transactions offering a higher fee take priority in the queue.
To prevent miners from producing unnecessarily large blocks (such as by padding to max size with zero value transactions) an excessive block size penalty is subtracted from the block reward\cite{CryptoNote}. This balances with the transaction fee reward to create a stable equilibrium of average block sizes.

% ------------------------------------------ %
\subsection{Outputs}
Units of a cryptocurrency can be split into very small fractional amounts though there is a finite limit called ``atomic units.'' For Monero the atomic unit size is \texttt{0.000000000001} XMR or one ``piconero''\cite{moneropedia}.

Because it would be impractical to track the movement of each of these atomic units individually, an ``output'' and ``input'' based system is used instead\cite{bitcoin_whitepaper}. These outputs combine amounts of currency into single units, similar to how a \$10 bill combines the value of 1000 pennies\cite{monero_outputs}. If no single output is large enough to cover a transaction they can be combined, however there are always at most two outputs per transaction comprising the payment amount and the ``change'' returned to the sender\cite{bitcoin_whitepaper}. The one drawback to this system is that, as outputs are ``locked'' while a transaction is taking place and pending sufficient confirmations, a user lacking multiple outputs will be unable to send another transaction until the first one is fully confirmed. This is a rare issue however as almost all frequent users will have multiple outputs and the input output system is managed invisibly to the end user by the wallet software\cite{monero_outputs}.

% ------------------------------------------ %
\subsection{Emission}
Both Monero and Wownero had no premine, instamine, or presale of any kind to ensure a fair and even distribution\cite{monero_about, wowbsite}. Monero had irregular emission with block rewards steadily decreasing until they reached a flat 0.6 XMR block reward which will continue in perpetuity\cite{monero_faq}. Wownero took a different approach with a total supply of 184,467,440 coins to be mined over ~50 years and no tail emission\cite{wowrepo}. This may be an issue long term as without an incentive to mine the security of the network could be reduced\cite{zero2monero}.

% ------------------------------------------ %
\subsection{Blockchain Upgrades}
Both Monero and Wownero utilize a ``hard fork'' mechanism to implement scheduled software upgrades into their networks\cite{wowrepo,monero_repo}. A hard fork occurs when a majority of nodes no longer accept older versions of the blockchain\cite{hard_fork}. In the case of Monero and Wownero this has only occurred due to official changes to implement new features such as new proof-of-work algorithms or implementation of ``bulletproofs''\cite{monero_repo}.

In a hard fork the blockchain becomes split into an old and new version of the blockchain\cite{hard_fork}. Because of this it is possible for an official hard fork to be rejected by nodes which do not accept the changes which leads to a split. This has happened with other cryptocurrencies such as ``Etherium Classic'' which was formed in 2016 due to differences of opinion on whether a smart contract hack should be rolled back\cite{eth_classic}. Monero and Wownero have prevented this from occurring (so far) through clear communication, ensuring there is broad favorable consensus regarding an upgrade before an update is released\cite{monero_upgrades}. Because only changes accepted almost unanimously are implemented a ``non-contentious hard-fork'' occurs in which the old blockchain dies off due to a lack of nodes running the older version.

Some changes, referred to as ``soft forks'', can occur while remaining backwards compatible with the original blockchain. These changes require only miner nodes to update, and normal nodes remain unaffected\cite{soft_fork}. 

To the end user all upgrades require only a simple software update of the client software\cite{monero_upgrades}.

% ------------------------------------------ %
\subsection{Consensus}
Because of the decentralized, peer-to-peer, node based nature of cryptocurrency, maintaining consensus between the distributed nodes of the network is of critical importance. A majority of nodes must agree on the same blockchain, block creation structure, and transaction rules in order to prevent malicious actors from exploiting the network. This is achieved through the use of consensus algorithms, which enable the nodes of the network to agree on a shared version of the blockchain.

To ensure every node uses the same blockchain, the chain with the highest cumulative difficulty\footnote{\textbf{Cumulative difficulty} refers to the total difficulty required to mine every block in the chain.} is considered to be the legitimate version\cite{zero2monero}. This ensures that, in order to force an alteration to the transaction history, an attacker must create a fork of the blockchain with higher total difficulty than the original. This would in almost all cases require the attacker to control over 50\% of the total network mining hashrate to grow faster than the main chain\cite{zero2monero}.

Both Monero and Wownero utilize a proof-of-work algorithm (described in a later section) to maintain block consensus\cite{monero_about,wowbsite}. Other cryptocurrencies may use proof-of-stake (in which block validators are chosen based on their stake in the network\cite{concensus_algorithms}), proof-of-capacity (in which miners prove storage capacity instead of computational power), proof-of-authority (in which validators are chosen based on their reputation or identity), or any other algorithm.
\pagebreak
