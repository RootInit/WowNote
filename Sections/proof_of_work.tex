\section{Proof-of-Work}
``Proof-of-work,'' or PoW, is a type of consensus algorithm used in blockchain networks since Bitcoin\cite{wikipedia_crypto}. In this concensus algorithm transactions are validated and new blocks are added to the blockchain through a computational puzzle that must be solved by miners\cite{bitcoin_whitepaper}. While PoW has some drawbacks, such as high energy consumption, it remains one of the most widely-used and secure consensus algorithms in blockchain technology\cite{concensus_algorithms}.

\subsection{Early Proof-of-Work}
Not all proof-of-work algorithms are equal, however. The first proof-of-work algorithm used was a double SHA-256\footnote{\textbf{SHA-256} is a cryptographic hash function used for data integrity and authenticity verification in various computer security applications.} based function used in Bitcoin\cite{bitcoin_whitepaper}. This was intended to result in an egalitarian ``one-CPU-one-vote'' system\cite{bitcoin_whitepaper}.

Unfortunately, and likely unforeseen to Satoshi Nakamoto, the explosion of Bitcoin's popularity resulted in bitcoin mining ASICs\footnote{An \textbf{ASIC} or Application-Specific Integrated Circuit is a computer chip that is designed to perform a specific function with high speed and efficiency due to its specialized nature\cite{glossary}.} being developed and advances in GPU compute technology made GPUs much more efficient at mining blocks. Because of this CPU mining was rendered uncompetitive and mostly impractical giving a massive advantage, by orders of magnitude, to owners of ASIC mining rigs and GPU clusters\cite{randomx_article}.

\subsection{CryptoNight}
The CryptoNote protocol on proposed an algorithm, later dubbed ``CryptoNight'' which was intended to ``close the gap between CPU (majority) and GPU/FPGA/ASIC (minority) miners''\cite{CryptoNote}. This algorithm worked by utilizing a memory bound function which would be most efficient on CPUs due to the large L3 cache modern CPUs possess\cite{CryptoNote}. The 2 Mb of memory required by the CryptoNight algorithm was thought to be extremely difficult to develop an ASIC around and GPU processors do not have internal cache but rather rely on significantly slower memory on a separate chip\cite{CryptoNote}.

Ultimately CryptoNight failed in its efforts to provide ASIC resistance and ASICs capable of running the CryptoNight algorithm were developed\cite{CryptoNight}. Coins utilizing CryptoNight were forced to either accepted ASIC mining or switch to a different PoW algorithm.

Choosing the latter option, the Monero Research Lab upgraded the CryptoNight algorithm to v7, v8, and ``CryptoNight-R'' though all of these modifications were eventually defeated by improvements in ASIC technology\cite{CryptoNight}. The CryptoNight-R algorithm introduced a novel change of randomized integer math which was \emph{temporarily} very effective but only used in Monero between March 9, 2019 and November 30th 2019 as a new superior algorithm was internally developed specifically for use with Monero\cite{monero_repo}.

\subsection{RandomX}
Based on the same core principle of introducing randomness as CryptoNight-R, the Monero Research Lab developed an entirely new Proof-of-Work algorithm called ``RandomX'' with much stronger ASIC resistance while retaining the same private hashing as CryptoNight\cite{randomx_article}. Rather than rely on any single specific attribute of CPUs (such as cache size), RandomX instead is based on execution of random code inside a virtual machine\cite{randomx_repo}. RandomX has two modes with different memory requirements. A \emph{light} mode requiring only 256 MiB of memory (intended for normal nodes) to use for verification and a \emph{fast} mode requiring 2080 MiB of memory (intended for miner nodes)\cite{randomx_repo}. Because of the \emph{random} nature of the algorithm, only general purpose processors are capable of efficiently running it rendering ASICs theoretically impossible\cite{randomx_article}. Additionally, the high memory requirement for mining makes the mining software easily detectable by antivirus programs, low memory IoT devices unable to mine at all, and prevents web applications from mining covertly. These restrictions should reduce botnet\footnote{A \textbf{botnet} is a network of infected computers that are remotely controlled by a single attacker for malicious purposes.} mining, making it much harder for a malicious actor to obtain a high percentage of hashrate illegitimatly\cite{randomx_medium}.

Monero implemented the RandomX algorithm on November 30th 2019 as part of v12 update\cite{monero_repo}.

\subsection{Wownero}
Wownero likewise implemented a series of ASIC resistant POW algorithms including a slightly modified version of CryptoNight-R on the October 6th 2018 update codenamed ``Cool Cage''\cite{wowrepo}. They also briefly used a pseudorandom function ``CryptoNight/WOW'' from February 19th 2019 until the implementation of, RandomX based, ``RandomWOW'' which, due to lower testing requirements, they were able to implement ahead of Monero on the June 14th 2019 codename ``F For Fappening'' hard fork\cite{wowrepo}.

RandomWoW's differs from RandomX in that it uses a scratchpad (workspace memory)\cite{randomx_medium} size of only 1 MB compared to RandomX's 2 MB scratchpad size\cite{wowrepo}. RandomWOW also has a limit of 16 chained VM executions per hash to further increase program compilation difficulty for GPUs and ASICS\cite{wowrepo}.

Wownero took an additional step to promote decentralization through preventing public mining pools by requiring ``Miner Block Header Signing'' which requires miners to sign blocks with their private key\cite{wowrepo,jwinterview}. Because private keys cannot be shared without giving away full access to your wallet, this rendered Wownero solo and private pool mining only as of the July 4th 2021, ``Junkie Jeff,'' update\cite{wowrepo}.
